\documentclass{article}
%\documentstyle[11pt,handout,psfig]{article}

\usepackage{fullpage,amssymb,amsmath,epsf}
\usepackage[12pt]{extsizes}

%These give really tight margins:
%\setlength{\topmargin}{-0.3in}
%\setlength{\textheight}{8.10in}
%\setlength{\textwidth}{5.8in}
%\setlength{\baselineskip}{0.1875in}
%\addtolength{\leftmargin}{-2.775in}
%\setlength{\footskip}{0.45in}
%\setlength{\oddsidemargin}{0.5in}
%\setlength{\evensidemargin}{0.5in}
%%\setlength{\headsep}{0pt}
%%\setlength{\headheight}{0pt}

%\setlength{\topmargin}{-0.5in}
\setlength{\textheight}{8in}
%\setlength{\textwidth}{5.0in}
%\setlength{\baselineskip}{0.1875in}
%\addtolength{\leftmargin}{-2.775in}
%\setlength{\footskip}{0.45in}
%\setlength{\oddsidemargin}{0.5in}
%\setlength{\evensidemargin}{0.5in}
%%\setlength{\headsep}{0pt}
%%\setlength{\headheight}{0pt}


\markright{XCS229i}
\pagestyle{myheadings}

\newcommand{\newsec}{\section}
\newcommand{\denselist}{\itemsep 0pt\partopsep 0pt}
\newcommand{\bitem}{\begin{itemize}\denselist}
\newcommand{\eitem}{\end{itemize}}
\newcommand{\benum}{\begin{enumerate}\denselist}
\newcommand{\eenum}{\end{enumerate}}

\newcommand{\fig}[1]{\private{\begin{center}
{\Large\bf ({#1})}
\end{center}}}

\newcommand{\cpsf}[1]{{\centerline{\psfig{#1}}}}
\newcommand{\mytitle}[1]{\centerline{\LARGE\bf #1}}

\newcommand{\myw}{{\bf w}}

\newcommand{\mypar}[1]{\vspace{1ex}\noindent{\bf {#1}}}

\def\thmcolon{\hspace{-.85em} {\bf :} }

\newtheorem{THEOREM}{Theorem}[section]
\newenvironment{theorem}{\begin{THEOREM} \thmcolon }%
                        {\end{THEOREM}}
\newtheorem{LEMMA}[THEOREM]{Lemma}
\newenvironment{lemma}{\begin{LEMMA} \thmcolon }%
                      {\end{LEMMA}}
\newtheorem{COROLLARY}[THEOREM]{Corollary}
\newenvironment{corollary}{\begin{COROLLARY} \thmcolon }%
                          {\end{COROLLARY}}
\newtheorem{PROPOSITION}[THEOREM]{Proposition}
\newenvironment{proposition}{\begin{PROPOSITION} \thmcolon }%
                            {\end{PROPOSITION}}
\newtheorem{DEFINITION}[THEOREM]{Definition}
\newenvironment{definition}{\begin{DEFINITION} \thmcolon \rm}%
                            {\end{DEFINITION}}
\newtheorem{CLAIM}[THEOREM]{Claim}
\newenvironment{claim}{\begin{CLAIM} \thmcolon \rm}%
                            {\end{CLAIM}}
\newtheorem{EXAMPLE}[THEOREM]{Example}
\newenvironment{example}{\begin{EXAMPLE} \thmcolon \rm}%
                            {\end{EXAMPLE}}
\newtheorem{REMARK}[THEOREM]{Remark}
\newenvironment{remark}{\begin{REMARK} \thmcolon \rm}%
                            {\end{REMARK}}
%\newenvironment{proof}{\noindent {\bf Proof:} \hspace{.677e\nexp}}%
%                      {}

%theorem
\newcommand{\thm}{\begin{theore\nexp}}
%lemma
\newcommand{\lem}{\begin{lemma}}
%proposition
\newcommand{\pro}{\begin{propositio\di}}
%definition
\newcommand{\dfn}{\begin{definitio\di}}
%remark
\newcommand{\rem}{\begin{remark}}
%example
\newcommand{\xam}{\begin{example}}
%corollary
\newcommand{\cor}{\begin{corollary}}
%proof
\newcommand{\prf}{\noindent{\bf Proof:} }
%end theorem
\newcommand{\ethm}{\end{theore\nexp}}
%end lemma
\newcommand{\elem}{\end{lemma}}
%end proposition
\newcommand{\epro}{\end{propositio\di}}
%end definition
\newcommand{\edfn}{\bbox\end{definitio\di}}
%end remark
\newcommand{\erem}{\bbox\end{remark}}
%end example
\newcommand{\exam}{\bbox\end{example}}
%end corollary
\newcommand{\ecor}{\end{corollary}}
%end proof
\newcommand{\eprf}{\bbox\vspace{0.1i\di}}
%begin equation
\newcommand{\beqn}{\begin{equatio\di}}
%end equation
\newcommand{\eeqn}{\end{equatio\di}}

%\newcommand{\eqref}[1]{Eq.~\ref{#1}}

\newcommand{\KB}{\mbox{\it KB\/}}
\newcommand{\infers}{\vdash}
\newcommand{\sat}{\models}
\newcommand{\bbox}{\vrule height7pt width4pt depth1pt}

\newcommand{\act}[1]{\stackrel{{#1}}{\rightarrow}}
\newcommand{\at}[1]{^{(#1)}}

\newcommand{\argmax}{{\rm argmax}}

\newcommand{\rimp}{\Rightarrow}
\newcommand{\dimp}{\Leftrightarrow}

\newcommand{\bX}{\mbox{\boldmath $X$}}
\newcommand{\bY}{\mbox{\boldmath $Y$}}
\newcommand{\bZ}{\mbox{\boldmath $Z$}}
\newcommand{\bU}{\mbox{\boldmath $U$}}
\newcommand{\bE}{\mbox{\boldmath $E$}}
\newcommand{\bx}{\mbox{\boldmath $x$}}
\newcommand{\be}{\mbox{\boldmath $e$}}
\newcommand{\by}{\mbox{\boldmath $y$}}
\newcommand{\bz}{\mbox{\boldmath $z$}}
\newcommand{\bu}{\mbox{\boldmath $u$}}
\newcommand{\bd}{\mbox{\boldmath $d$}}
\newcommand{\smbx}{\mbox{\boldmath $\scriptstyle x$}}
\newcommand{\smbd}{\mbox{\boldmath $\scriptstyle d$}}
\newcommand{\smby}{\mbox{\boldmath $\scriptstyle y$}}
\newcommand{\smbe}{\mbox{\boldmath $\scriptstyle e$}}

\newcommand{\Parents}{\mbox{\it Parents\/}}
\newcommand{\B}{{\cal B}}
\newcommand{\calH}{{\cal H}}

\newcommand{\word}[1]{\mbox{\it #1\/}}
\newcommand{\Action}{\word{Actio\di}}
\newcommand{\Proposition}{\word{Propositio\di}}
\newcommand{\true}{\word{true}}
\newcommand{\false}{\word{false}}
\newcommand{\Pre}{\word{Pre}}
\newcommand{\Add}{\word{Add}}
\newcommand{\Del}{\word{Del}}
\newcommand{\Result}{\word{Result}}
\newcommand{\Regress}{\word{Regress}}
\newcommand{\Maintain}{\word{Maintai\di}}

\newcommand{\bor}{\bigvee}
\newcommand{\invert}[1]{{#1}^{-1}}

\newcommand{\commentout}[1]{}

\newcommand{\bmu}{\mbox{\boldmath $\mu$}}
\newcommand{\btheta}{\mbox{\boldmath $\theta$}}
\newcommand{\IR}{\mbox{$I\!\!R$}}

\newcommand{\tval}[1]{{#1}^{1}}
\newcommand{\fval}[1]{{#1}^{0}}

\newcommand{\tr}{{\rm tr}}
\newcommand{\vecy}{{\vec{y}}}
\renewcommand{\Re}{{\mathbb R}}

\def\twofigbox#1#2{%
\noindent\begin{minipage}{\textwidth}%
\epsfxsize=0.35\maxfigwidth
\noindent \epsffile{#1}\hfill
\epsfxsize=0.35\maxfigwidth
\epsffile{#2}\\
\makebox[0.35\textwidth]{(a)}\hfill\makebox[0.35\textwidth]{(b)}%
\end{minipage}}

\def\twofigboxcd#1#2{%
\noindent\begin{minipage}{\textwidth}%
\epsfxsize=0.35\maxfigwidth
\noindent \epsffile{#1}\hfill
\epsfxsize=0.35\maxfigwidth
\epsffile{#2}\\
\makebox[0.35\textwidth]{(c)}\hfill\makebox[0.35\textwidth]{(d)}%
\end{minipage}}

\def\twofigboxnolabel#1#2{%
\begin{minipage}{\textwidth}%
\epsfxsize=0.35\maxfigwidth
\noindent \epsffile{#1}\hfill
\epsfxsize=0.35\maxfigwidth
\epsffile{#2}\\
%\makebox[0.48\textwidth]{(a)}\hfill\makebox[0.48\textwidth]{(b)}%
\end{minipage}
}

\def\twofigboxnolabelFive#1#2{%
\begin{minipage}{\textwidth}%
\hbox to 0.5in{}\epsfxsize=0.35\maxfigwidth
\noindent \epsffile{#1}\hfill
\epsfxsize=0.35\maxfigwidth
\epsffile{#2}\hbox to 0.5in{}\\
%\makebox[0.48\textwidth]{(a)}\hfill\makebox[0.48\textwidth]{(b)}%
\end{minipage}
}

\def\threefigbox#1#2#3{%
\noindent\begin{minipage}{\textwidth}%
\epsfxsize=0.33\maxfigwidth
\noindent \epsffile{#1}\hfill
\epsfxsize=0.33\maxfigwidth
\noindent \epsffile{#2}\hfill 
\epsfxsize=0.33\maxfigwidth
\epsffile{#3}\\
\makebox[0.31\textwidth]{{\scriptsize (a)}}\hfill%
\makebox[0.31\textwidth]{{\scriptsize (b)}}\hfill
\makebox[0.31\textwidth]{{\scriptsize (c)}}%
\smallskip
\end{minipage}}

\def\threefigboxnolabel#1#2#3{%
\noindent\begin{minipage}{\textwidth}%
\epsfxsize=0.33\maxfigwidth
\noindent \epsffile{#1}\hfill
\epsfxsize=0.33\maxfigwidth
\noindent \epsffile{#2}\hfill 
\epsfxsize=0.33\maxfigwidth
\epsffile{#3}\\
%\makebox[0.31\textwidth]{{\scriptsize (a)}}\hfill%
%\makebox[0.31\textwidth]{{\scriptsize (b)}}\hfill
%\makebox[0.31\textwidth]{{\scriptsize (c)}}%
%\smallskip
\end{minipage}}

\newlength{\maxfigwidth}
\setlength{\maxfigwidth}{\textwidth}
%\def\captionsize {\footnotesize}
\def\captionsize {}

\newcommand{\xsi}{{x^{(i)}}}
\newcommand{\xsd}{{x^{(d)}}}
\newcommand{\xsj}{{x^{(j)}}}
\newcommand{\ysi}{{y^{(i)}}}
\newcommand{\ysj}{{y^{(j)}}}
\newcommand{\gsi}{{\gamma^{(i)}}}
\newcommand{\wsi}{{w^{(i)}}}
\newcommand{\esi}{{\epsilon^{(i)}}}
\newcommand{\calN}{{\cal N}}
\newcommand{\calX}{{\cal X}}
\newcommand{\calY}{{\cal Y}}
\newcommand{\calL}{{\cal L}}
\newcommand{\calP}{{\cal P}}
\newcommand{\calD}{{\cal D}}
\newcommand{\ytil}{{\tilde{y}}}

\newcommand{\Ber}{{\rm Bernoulli}}
\newcommand{\E}{{\rm E}}

\newcommand{\pstar}{{p^{\ast}}}
\newcommand{\bstar}{{b^{\ast}}}
\newcommand{\dstar}{{d^{\ast}}}
\newcommand{\wstar}{{w^{\ast}}}
\newcommand{\alphastar}{\alpha^{\ast}}
\newcommand{\alphastari}{{\alpha_i^{\ast}}}
\newcommand{\betastar}{{\beta^{\ast}}}
\newcommand{\tol}{{\textit tol}}
\newcommand{\phihat}{\hat\phi}
\newcommand{\ehat}{\hat\varepsilon}
\newcommand{\hhat}{\hat{h}}
\newcommand{\hstar}{h^\ast}
\newcommand{\VC}{{\rm VC}}

\newcommand{\hwb}{{h_{w,b}}}

\usepackage{graphicx}

%\renewcommand{\epsffile}[1]{
%	\includegraphics[width=\epsfxsize]{#1}
%}

\newcommand{\di}{{d}}
\newcommand{\nexp}{{n}}
\newcommand{\vcd}{{\textbf{D}}}

\begin{document}
\title{XCS229i Lecture Notes}
\author{Andrew Ng}
\date{}
\maketitle

\setcounter{part}{9}
\part{Factor analysis}

When we have data $\xsi\in \Re^\di$ that comes from a mixture of several Gaussians,
the EM algorithm can be applied to fit a mixture model.  In this
setting, we usually imagine problems where we have sufficient data
to be able to discern the multiple-Gaussian structure in the data.  For instance,
this would be the case if our training set size $\nexp$ was significantly
larger than the dimension $\di$ of the data.

Now, consider a setting in which $\di \gg \nexp$.
In such a problem, it might be difficult to model
the data even with a single Gaussian, much less a mixture of Gaussian. Specifically,
since the $\nexp$ data points span only a low-dimensional subspace of $\Re^\di$, if we model
the data as Gaussian, and estimate the mean and covariance using the usual
maximum likelihood estimators,
\begin{eqnarray*}
\mu &=& \frac{1}{\nexp}\sum_{i=1}^\nexp \xsi  \\
\Sigma &=& \frac{1}{\nexp}\sum_{i=1}^\nexp (\xsi -\mu) (\xsi -\mu)^T,
\end{eqnarray*}
we would find that the matrix $\Sigma$ is singular.  This means that $\Sigma^{-1}$
does not exist, and $1/|\Sigma|^{1/2} = 1/0$. But both of these terms are needed
in computing the usual density of a multivariate Gaussian distribution.  Another way
of stating this difficulty is that maximum likelihood estimates of the parameters
result in a Gaussian that places all of its probability in the affine space spanned
by the data,\footnote{This is the set of points $x$ satisfying $x = \sum_{i=1}^\nexp \alpha_i \xsi$,
for some $\alpha_i$'s so that $\sum_{i=1}^\nexp \alpha_1 = 1$.} and this corresponds to a singular
covariance matrix.

More generally, unless $\nexp$ exceeds $\di$ by some reasonable amount, the maximum likelihood
estimates of the mean and covariance may be quite poor.  Nonetheless, we would still like to
be able to fit a reasonable Gaussian model to the data, and perhaps capture some
interesting covariance structure in the data.  How can we do this?

In the next section, we begin by reviewing two possible restrictions on $\Sigma$
that allow us to fit $\Sigma$ with small amounts of data but neither will give a
satisfactory solution to our problem.  We next discuss some
properties of Gaussians that will be needed later; specifically, how to find marginal
and conditonal distributions of Gaussians.  Finally, we present the factor analysis model,
and EM for it.


\section{Restrictions of $\Sigma$}

If we do not have sufficient data to fit a full covariance matrix, we may place some
restrictions on the space of matrices $\Sigma$ that we will consider.  For instance,
we may
choose to fit a covariance matrix $\Sigma$ that is diagonal.  In this setting, the reader
may easily verify that the maximum likelihood estimate of the covariance matrix is given by the
diagonal matrix $\Sigma$ satisfying
\[
\Sigma_{jj} = \frac{1}{\nexp}\sum_{i=1}^\nexp (x_j^{(i)} -\mu_j)^2.
\]
Thus, $\Sigma_{jj}$ is just the empirical estimate of the variance of the $j$-th coordinate
of the data.

Recall that the contours of a Gaussian density are ellipses.  A diagonal $\Sigma$ corresponds
to a Gaussian where the major axes of these ellipses are axis-aligned.

Sometimes, we may place a further restriction on the covariance matrix that not only
must it be diagonal, but its diagonal entries must all be equal.  In this setting,
we have $\Sigma = \sigma^2I$, where $\sigma^2$ is the parameter under our control.
The maximum likelihood estimate of $\sigma^2$ can be found to be:
\[
\sigma^2 = \frac{1}{\nexp \di} \sum_{j=1}^\di \sum_{i=1}^\nexp (x_j^{(i)} -\mu_j)^2.
\]
This model corresponds to using Gaussians whose densities have contours that are
circles (in 2 dimensions; or spheres/hyperspheres in higher dimensions).

If we are fitting a full, unconstrained, covariance matrix $\Sigma$ to data, it is
necessary that $\nexp \geq \di+1$ in order for the maximum likelihood estimate of $\Sigma$ not
to be singular.  Under either of the two restrictions above, we may obtain non-singular
$\Sigma$ when $\nexp \geq 2$.

However, restricting $\Sigma$ to be diagonal also means modeling the different coordinates
$x_i$, $x_j$ of the data as being uncorrelated and independent.  Often, it would be nice to
be able to capture some interesting correlation structure in the data.  If we were to use
either of the restrictions on $\Sigma$ described above, we would therefore
fail to do so.  In this set of notes, we will describe the factor analysis model,
which uses more parameters than the diagonal $\Sigma$ and captures some correlations
in the data, but also without having to fit a full covariance matrix.

\section{Marginals and conditionals of Gaussians}

Before describing factor analysis, we digress to talk about how to find conditional
and marginal distributions of random variables with a joint multivariate Gaussian distribution.

Suppose we have a vector-valued random variable
\[
x = \left[\begin{array}{c} x_1 \\ x_2 \end{array} \right],
\]
where $x_1 \in \Re^r$, $x_2 \in \Re^s$, and $x \in \Re^{r+s}$.  Suppose
$x \sim \calN(\mu, \Sigma)$, where
\[
\mu = \left[\begin{array}{c} \mu_1 \\ \mu_2 \end{array} \right],
\quad
\Sigma = \left[\begin{array}{cc} \Sigma_{11} & \Sigma_{12} \\ \Sigma_{21} & \Sigma_{22} \end{array} \right].
\]
Here, $\mu_1 \in \Re^r$, $\mu_2 \in \Re^s$, $\Sigma_{11} \in \Re^{r\times r}$,
$\Sigma_{12} \in \Re^{r\times s}$, and so on.  Note that since covariance matrices
are symmetric, $\Sigma_{12} = \Sigma_{21}^T$.

Under our assumptions, $x_1$ and $x_2$ are jointly multivariate Gaussian.  What is the
marginal distribution of $x_1$?  It is not hard to see that $\E[x_1] = \mu_1$, and
that $\Cov(x_1) = \E[(x_1-\mu_1)(x_1-\mu_1)] = \Sigma_{11}$.  To see that the latter
is true, note that by definition of the joint covariance of $x_1$ and $x_2$, we have
that
\begin{eqnarray*}
\Cov(x) &=& \Sigma \\
&=& \left[\begin{array}{cc} \Sigma_{11} & \Sigma_{12} \\ \Sigma_{21} & \Sigma_{22} \end{array} \right] \\
&=& \E[(x-\mu)(x-\mu)^T] \\
&=& \E\left[
\left(\begin{array}{c} x_1 - \mu_1 \\ x_2 - \mu_2 \end{array} \right)
\left(\begin{array}{c} x_1 - \mu_1 \\ x_2 - \mu_2 \end{array} \right)^T \right] \\
&=& \E\left[
\begin{array}{cc}
(x_1 - \mu_1)(x_1-\mu_1)^T & (x_1 - \mu_1)(x_2-\mu_2)^T \\
(x_2 - \mu_2)(x_1-\mu_1)^T & (x_2 - \mu_2)(x_2-\mu_2)^T
\end{array}
\right].
\end{eqnarray*}
Matching the upper-left subblocks in the matrices in the second and the last lines above
gives the result.

Since marginal distributions of Gaussians are themselves Gaussian, we therefore have
that the marginal distribution of $x_1$ is given by $x_1 \sim \calN(\mu_1, \Sigma_{11})$.

Also, we can ask, what is the conditional distribution of $x_1$ given $x_2$?
By referring to the definition of the multivariate Gaussian distribution, it
can be shown that $x_1 | x_2 \sim \calN(\mu_{1|2}, \Sigma_{1|2})$, where

\begin{eqnarray}
\mu_{1|2} &=& \mu_1 + \Sigma_{12} \Sigma_{22}^{-1} (x_2 - \mu_2), \label{eqn-gausscond1} \\
\Sigma_{1|2} &=& \Sigma_{11} - \Sigma_{12}\Sigma_{22}^{-1}\Sigma_{21}.\label{eqn-gausscond2}
\end{eqnarray}

When we work with the factor analysis model in the next section, these formulas for
finding conditional and marginal distributions of Gaussians will be very useful.

\section{The Factor analysis model}

In the factor analysis model, we posit a joint distribution on $(x,z)$ as follows, where $z \in \Re^k$
is a latent random variable:
\begin{eqnarray*}
z &\sim& \calN(0,I) \\
x|z &\sim& \calN(\mu+\Lambda z, \Psi).
\end{eqnarray*}

Here, the parameters of our model are the vector $\mu \in \Re^\di$, the matrix
$\Lambda \in \Re^{\di\times k}$, and the diagonal matrix $\Psi \in \Re^{\di\times \di}$.
The value of $k$ is usually chosen to be smaller than $\di$.

Thus, we imagine that each datapoint $\xsi$ is generated by sampling a $k$ dimension
multivariate Gaussian $\zsi$.  Then, it is mapped to a $\di$-dimensional affine space of
$\Re^\di$ by computing $\mu+\Lambda \zsi$.  Lastly, $\xsi$ is generated by adding
covariance $\Psi$ noise to $\mu+\Lambda \zsi$.

Equivalently (convince yourself that this is the case), we can therefore also define
the factor analysis model according to
\begin{eqnarray*}
z &\sim& \calN(0,I) \\
\epsilon &\sim& \calN(0, \Psi) \\
x &=& \mu+\Lambda z + \epsilon
\end{eqnarray*}
where $\epsilon$ and $z$ are independent.

Let's work out exactly what distribution our model defines.  Our random
variables $z$ and $x$ have a joint Gaussian distribution
\[
\stackvec{z}{x} \sim \calN(\mu_{zx}, \Sigma).
\]
We will now find $\mu_{zx}$ and $\Sigma$.

We know that $\E[z] = 0$, from the fact that $z \sim \calN(0,I)$.  Also, we have
that
\begin{eqnarray*}
\E[x] &=& \E[\mu+\Lambda z +\epsilon] \\
  &=& \mu + \Lambda \E[z] + \E[\epsilon] \\
  &=& \mu.
\end{eqnarray*}
Putting these together, we obtain
\[
\mu_{zx} = \stackvec{\vec{0}}{\mu}
\]
Next, to find $\Sigma$, we need to calculate
$\Sigma_{zz} = \E[(z-\E[z])(z-\E[z])^T]$ (the upper-left block of $\Sigma$),
$\Sigma_{zx} = \E[(z-\E[z])(x-\E[x])^T]$ (upper-right block),
and $\Sigma_{xx} = \E[(x-\E[x])(x-\E[x])^T]$ (lower-right block).

Now, since $z \sim \calN(0,I)$, we easily find that $\Sigma_{zz} = \Cov(z) = I$.
Also,
\begin{eqnarray*}
\E[(z-\E[z])(x-\E[x])^T] &=& \E[z(\mu+\Lambda z + \epsilon - \mu)^T] \\
  &=& \E[zz^T]\Lambda^T + \E[z\epsilon^T] \\
  &=& \Lambda^T.
\end{eqnarray*}
In the last step, we used the fact that $\E[zz^T] = \Cov(z)$ (since $z$ has zero mean),
and $\E[z\epsilon^T] = \E[z]\E[\epsilon^T] = 0$ (since $z$ and $\epsilon$ are independent,
and hence the expectation of their product is the product of their expectations).
Similarly, we can find $\Sigma_{xx}$ as follows:
\begin{eqnarray*}
\E[(x-\E[x])(x-\E[x])^T] &=& \E[(\mu+\Lambda z + \epsilon - \mu) (\mu+\Lambda z + \epsilon - \mu)^T] \\
&=& \E[\Lambda zz^T \Lambda^T + \epsilon z^T\Lambda^T + \Lambda z \epsilon^T + \epsilon\epsilon^T] \\
&=& \Lambda \E[zz^T] \Lambda^T + \E[\epsilon\epsilon^T] \\
&=& \Lambda \Lambda^T + \Psi.
\end{eqnarray*}
Putting everything together, we therefore have that
\begin{equation}
\stackvec{z}{x} \sim \calN\left(\stackvec{\vec{0}}{\mu}, \stackmat{I}{\Lambda^T}{\Lambda}{\Lambda\Lambda^T+\Psi}\right). \label{eqn-jointzx}
\end{equation}

Hence, we also see that the marginal distribution of $x$ is given
by $x \sim \calN(\mu, \Lambda\Lambda^T+\Psi)$.  Thus, given a training set $\{\xsi; i=1,\ldots,\nexp\}$,
we can write down the log likelihood of the parameters:
\[
\ell(\mu, \Lambda, \Psi) = \log \prod_{i=1}^\nexp \frac{1}{(2\pi)^{\di/2}|\Lambda\Lambda^T+\Psi|^{1/2}}
\exp\left(-\frac{1}{2} (\xsi - \mu)^T (\Lambda\Lambda^T+\Psi)^{-1} (\xsi-\mu) \right).
\]
To perform maximum likelihood estimation, we would like to maximize this quantity with
respect to the parameters.  But maximizing this formula explicitly is hard (try it yourself),
and we are aware of no algorithm that does so in closed-form.
So, we will instead use to the EM algorithm.  In the next section, we derive EM for
factor analysis.

\section{EM for factor analysis}

The derivation for the E-step is easy.  We need to
compute $Q_i(\zsi) = p(\zsi | \xsi; \mu, \Lambda, \Psi)$.  By substituting the
distribution given in Equation~(\ref{eqn-jointzx}) into
the formulas~(\ref{eqn-gausscond1}-\ref{eqn-gausscond2}) used for finding the conditional
distribution of a Gaussian, we
find that $\zsi | \xsi; \mu, \Lambda, \Psi \sim \calN(\mu_{\zsi|\xsi}, \Sigma_{\zsi|\xsi})$,
where
\begin{eqnarray*}
\mu_{\zsi|\xsi} &=& \Lambda^T(\Lambda\Lambda^T+\Psi)^{-1}(\xsi-\mu), \\
\Sigma_{\zsi|\xsi} &=& I - \Lambda^T(\Lambda\Lambda^T+\Psi)^{-1}\Lambda.
\end{eqnarray*}
So, using these definitions for $\mu_{\zsi|\xsi}$ and $\Sigma_{\zsi|\xsi}$, we have
\[
Q_i(\zsi) = \frac{1}{(2\pi)^{k/2}|\Sigma_{\zsi|\xsi}|^{1/2}} \exp\left(-\frac{1}{2}(\zsi-\mu_{\zsi|\xsi})^T\Sigma_{\zsi|\xsi}^{-1}(\zsi-\mu_{\zsi|\xsi})\right).
\]

Let's now work out the M-step.  Here, we need to maximize
\begin{equation}
\sum_{i=1}^\nexp \int_\zsi Q_i(\zsi) \log \frac{p(\xsi, \zsi ; \mu, \Lambda, \Psi)}{Q_i(\zsi)} d\zsi
\label{eqn-emfa1}
\end{equation}
with respect to the parameters $\mu, \Lambda, \Psi$.  We will work out only the
optimization with respect to $\Lambda$, and leave the derivations
of the updates for $\mu$ and $\Psi$ as an exercise to the reader.

We can simplify Equation~(\ref{eqn-emfa1}) as follows:
\begin{eqnarray}
&& \hbox to -0.5in{} \sum_{i=1}^\nexp \int_\zsi Q_i(\zsi) \left[ \log p(\xsi | \zsi ; \mu, \Lambda, \Psi) + \log p(\zsi)  - \log Q_i(\zsi) \right] d\zsi  \\
&=& \sum_{i=1}^\nexp \E_{\zsi \sim Q_i} \left[
\log p(\xsi| \zsi ; \mu, \Lambda, \Psi) + \log p(\zsi)
- \log Q_i(\zsi) \right]
\end{eqnarray}
Here, the ``$\zsi \sim Q_i$'' subscript indicates that the expectation is with respect to
$\zsi$ drawn from $Q_i$.  In the subsequent development,
we will omit this subscript when there is no risk of ambiguity.
Dropping terms that do not depend on the
parameters, we find that we need to maximize:
\begin{eqnarray*}
&&\hbox to -0.4in{} \sum_{i=1}^\nexp \E\left[ \log p(\xsi| \zsi ; \mu, \Lambda, \Psi)\right] \\
&=& \sum_{i=1}^\nexp \E\left[ \log \frac{1}{(2\pi)^{\di/2}|\Psi|^{1/2}} \exp\left(-\frac{1}{2}
(\xsi-\mu-\Lambda \zsi)^T\Psi^{-1} (\xsi-\mu-\Lambda \zsi)\right)\right] \\
&=& \sum_{i=1}^\nexp \E\left[ -\frac{1}{2} \log|\Psi| - \frac{n}{2}\log (2\pi) - \frac{1}{2}
(\xsi-\mu-\Lambda \zsi)^T\Psi^{-1} (\xsi-\mu-\Lambda \zsi)\right]
\end{eqnarray*}
Let's maximize this with respect to $\Lambda$. Only the last term above depends on $\Lambda$.
Taking derivatives, and using the facts that $\tr\, a = a$ (for $a \in \Re$), $\tr AB = \tr BA$,
and $\nabla_A \tr ABA^TC = CAB+C^TAB^T$, we get:
\begin{eqnarray*}
&& \hbox to -0.4in{} \nabla_\Lambda \sum_{i=1}^\nexp - \E\left[\frac{1}{2} (\xsi-\mu-\Lambda \zsi)^T\Psi^{-1} (\xsi-\mu-\Lambda \zsi) \right]\\
&=& \sum_{i=1}^\nexp \nabla_\Lambda \E\left[ - \tr \frac{1}{2}\zsi^T \Lambda^T \Psi^{-1} \Lambda \zsi + \tr \zsi^T\Lambda^T \Psi^{-1} (\xsi-\mu) \right]\\
&=& \sum_{i=1}^\nexp \nabla_\Lambda \E\left[  - \tr \frac{1}{2}\Lambda^T \Psi^{-1} \Lambda \zsi \zsi^T + \tr \Lambda^T \Psi^{-1} (\xsi-\mu) \zsi^T \right]\\
&=& \sum_{i=1}^\nexp \E\left[ - \Psi^{-1} \Lambda \zsi \zsi^T + \Psi^{-1} (\xsi-\mu) \zsi^T \right]
\end{eqnarray*}
Setting this to zero and simplifying, we get:
\[
\sum_{i=1}^\nexp \Lambda \E_{\zsi \sim Q_i}\left[ \zsi \zsi^T\right] =
\sum_{i=1}^\nexp (\xsi-\mu) \E_{\zsi \sim Q_i}\left[\zsi^T \right].
\]
Hence, solving for $\Lambda$, we obtain
\begin{equation}
\Lambda =
\left(\sum_{i=1}^\nexp (\xsi-\mu) \E_{\zsi \sim Q_i}\left[\zsi^T \right]\right)
\left(\sum_{i=1}^\nexp \E_{\zsi \sim Q_i} \left[ \zsi \zsi^T\right]\right)^{-1}.
\label{eqn-gammaupdate}
\end{equation}
It is interesting to note the close relationship between this equation and
the normal equation that we'd derived for least squares regression,
\[
\hbox{``$\theta^T = (y^TX)(X^TX)^{-1}$.''}
\]
The analogy is that here, the $x$'s are a linear function of the $z$'s (plus noise).  Given
the ``guesses'' for $z$ that the E-step has found, we will now try to estimate
the unknown linearity $\Lambda$ relating the $x$'s and $z$'s.  It is therefore no
surprise that we obtain something similar to the normal equation.  There is, however,
one important difference between this and an algorithm that performs least squares using
just the ``best guesses'' of the $z$'s; we will see this difference shortly.

To complete our M-step update, let's work out the values of the expectations in Equation~(\ref{eqn-gammaupdate}).
From our definition of $Q_i$ being Gaussian with mean $\mu_{\zsi|\xsi}$ and covariance $\Sigma_{\zsi|\xsi}$, we
easily find
\begin{eqnarray*}
\E_{\zsi \sim Q_i}\left[\zsi^T \right] &=& \mu_{\zsi|\xsi}^T \\
\E_{\zsi \sim Q_i} \left[ \zsi \zsi^T\right] &=&
\mu_{\zsi|\xsi} \mu_{\zsi|\xsi}^T + \Sigma_{\zsi|\xsi}.
\end{eqnarray*}
The latter comes from the fact that, for a random variable $Y$,
$\Cov(Y) = \E[YY^T] - \E[Y]\E[Y]^T$, and hence
$\E[YY^T] = \E[Y]\E[Y]^T + \Cov(Y)$.
Substituting this back into Equation~(\ref{eqn-gammaupdate}), we get the M-step update
for $\Lambda$:
\begin{equation}
\Lambda =
\left(\sum_{i=1}^\nexp (\xsi-\mu) \mu_{\zsi|\xsi}^T\right)
\left(\sum_{i=1}^\nexp \mu_{\zsi|\xsi} \mu_{\zsi|\xsi}^T + \Sigma_{\zsi|\xsi}\right)^{-1}.
\end{equation}
It is important to note the presence of the $\Sigma_{\zsi|\xsi}$ on the right
hand side of this equation.  This is the covariance in the posterior distribution
$p(\zsi|\xsi)$ of $\zsi$ give $\xsi$, and the M-step must take into account this uncertainty
about $\zsi$ in the posterior. A common mistake in deriving EM is to assume that in the E-step,
we need to calculate only expectation $E[z]$ of the latent random variable $z$,
and then plug that into the optimization in the M-step everywhere $z$ occurs.
While this worked for simple problems such as the mixture of Gaussians, in our
derivation for factor analysis,
we needed $E[zz^T]$ as well $\E[z]$; and as we saw,
$E[zz^T]$ and $\E[z]\E[z]^T$ differ by the quantity $\Sigma_{z|x}$.
%Thus, the EM
Thus, the M-step update must take into account the covariance of $z$
in the posterior distribution $p(\zsi|\xsi)$.

Lastly, we can also find the M-step optimizations for the parameters $\mu$ and $\Psi$.
It is not hard to show that the first is given by
\[
\mu = \frac{1}{\nexp}\sum_{i=1}^\nexp \xsi.
\]
Since this doesn't change as the parameters are varied (i.e., unlike the update for $\Lambda$, the
right hand side
does not depend on $Q_i(\zsi) = p(\zsi|\xsi; \mu, \Lambda, \Psi)$, which in turn depends on the parameters),
this can be calculated just once and needs not be further updated as the algorithm is
run.  Similarly, the diagonal $\Psi$ can be found by calculating
\[
\Phi = \frac{1}{\nexp} \sum_{i=1}^\nexp \xsi \xsi^T -
\xsi \mu_{\zsi|\xsi}^T \Lambda^T
- \Lambda \mu_{\zsi|\xsi} \xsi^T +
\Lambda (\mu_{\zsi|\xsi} \mu_{\zsi|\xsi}^T + \Sigma_{\zsi|\xsi})\Lambda^T,
\]
and setting $\Psi_{ii} = \Phi_{ii}$ (i.e., letting $\Psi$ be the diagonal matrix
containing only the diagonal entries of $\Phi$).




%\begin{center}
%\epsfxsize=3in
%\epsffile{foo.eps}
%\end{center}

\end{document}


